\documentclass{article}
\usepackage{xeCJK}
\usepackage{amsmath}


\setCJKmainfont{Microsoft YaHei}
\linespread{1.5}


\begin{document}
一:\\
题目1:\\
使用bellman-ford算法,进行k次松弛操作即可限制边的数量,且该算法可以处理负权值边
和k次条件下的负权值环。
题目2:\\
重新计算边的权值,在原有权值上加上边的终点的停留时间,即
\[
w(u, v) = length(u, v) + time(v)
\]
图上无负权值边且为单源最短路径,调用dijistra算法即可求出结果。
\newpage
三:\\
构建图模型,将数字作为顶点,若两个数字满足“素数伴侣”条件,就在这两个顶点之间
加上一条边,求出所得图的最大匹配数即为“最佳方案”的“素数伴侣”对数。
\end{document}