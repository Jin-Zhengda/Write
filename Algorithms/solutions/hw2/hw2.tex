\documentclass{article}
\usepackage{xeCJK}
\usepackage{algorithm}
\usepackage{algorithmic}

\setCJKmainfont{SimSun} 
\renewcommand{\algorithmicrequire}{\textbf{Input:}}
\renewcommand{\algorithmicensure}{\textbf{Output:}}
\linespread{1.2}


\begin{document}
\setlength{\parindent}{0pt}

一:
\\
递归深度为 $\log n$,每层需要 $n$ 的时间,因此总的时间复杂度为 $O(n \log n)$。
\\
二:
\begin{algorithm}
    \caption{\textbf{binary\_search}}
    \label{alg:search}
    \begin{algorithmic}[1]
        \REQUIRE array sorted in ascending order, target, lo, hi
        \ENSURE index of the target if found, -1 if not found
        \STATE $mid = (lo + hi) // 2$
        \WHILE{$lo \leq hi$}
        \IF{$arr[mid] == target$}
        \STATE \textbf{return} $mid$
        \ELSIF{$arr[mid] < target$}
        \STATE \textbf{return} $binary\_search(arr, target, mid + 1, hi)$
        \ELSE
        \STATE \textbf{return} $binary\_search(arr, target, lo, mid - 1)$
        \ENDIF
        \ENDWHILE
        \STATE \textbf{return} $-1$
    \end{algorithmic}
\end{algorithm}

上述算法采取二分查找的策略,每次查找比较数组中间的元素,
如果中间元素大于目标值,则目标值在数组的前半部分,对前半部分进行递归调用,
否则在后半部分, 对后半部分进行递归调用。

该算法的时间复杂度的递归方程为
$$
T(n) = T(\frac{n}{2}) + 1
$$
由 Master定理, $1 = \Theta(n^{\log_{2}1}) = \Theta(1)$, 
故 $T(n) = \Theta(n^{\log_{2}1}\log n) = \Theta(\log n)$ , 
即此算法的时间复杂度为 $\Theta(\log n)$。 \\

\newpage
三:\\
1.在 $S$ 中找出其中位数 $median$ 。 \\
2.将 $S$ 中每个数与 $median$ 作差取绝对值,得到数组 $d$。\\
3.在 $d$ 中找出第 $k$ 个小的数 $kmin$。 \\
4.遍历 d, 找出小于等于 $kmin$ 对应的下标 $i$, 则所有的 $S[i]$ 构成最接近 $median$ 的 $k$ 个数。\\
伪代码如下:

\begin{algorithm}
    \caption{\textbf{Search for k numbers closest to the median of S}}
    \label{alg:search_k_median}
    \begin{algorithmic}[1]
        \REQUIRE $S$, $k$
        \ENSURE $k$ numbers in $S$ that are closest to the median of $S$
        \STATE $n = len(S)$
        \STATE $median = Select(S, n // 2)$
        \STATE $d = [\; ]$
        \FOR{$i = 0$ to $n - 1$}
        \STATE $d[i] = abs(S[i] - median)$
        \ENDFOR
        \STATE $kmin = Select(d, k)$
        \STATE $returnS = [\; ]$
        \FOR{$i = 0$ to $n - 1$}
        \IF{$d[i] \leq kmin$}
        \STATE $returnS.append(S[i])$
        \ENDIF 
        \ENDFOR
        \STATE \textbf{return} $returnS$
    \end{algorithmic}
\end{algorithm}

\newpage
下面是 $Select$ 算法的伪代码,实现在 $O(n)$ 的时间复杂度下找出一组数中第 $k$ 个小的数。 
\begin{algorithm}
    \caption{\textbf{find the k-th smallest number in arr}}
    \label{alg:search_k-th_small}
    \begin{algorithmic}[1]
        \REQUIRE $arr$, $k$
        \ENSURE the $k-th$ smallest number in $arr$
        \STATE $n = len(arr)$
        \FOR{$i = 0$ to $n / 5$}
        \STATE $insert\_sort(arr[(i - 1)*5 + 1:\; (i - 1)*5 + 5])$
        \STATE $swap(arr[i], arr[(i -1)*5 + 3])$
        \ENDFOR
        \STATE $x = Select(arr[: n // 5], n / 10)$
        \STATE $index = partition(arr, x)$
        \IF{$k == index$}
        \STATE \textbf{return} $x$
        \ELSIF{$k < index$}
        \STATE \textbf{return} $Select(arr[: index -1 ], k)$
        \ELSE 
        \STATE \textbf{return} $Select(arr[index + 1:], k - index)$      
        \ENDIF  
    \end{algorithmic}
\end{algorithm}

\newpage
下面是 $partition$ 算法的伪代码,实现在 $O(n)$ 的时间复杂度下将数组进行划分。
\begin{algorithm}
    \caption{\textbf{partition the arr}}
    \label{alg:partition}
    \begin{algorithmic}[1]
        \REQUIRE $arr$, $x$
        \ENSURE the index of $x$
        \STATE $n = len(arr)$
        \STATE $i = 0$
        \FOR{$j = 1$ to $n - 1$}
        \IF{$arr[j] \leq x$}
        \STATE $i = i + 1$
        \STATE $swap(arr[i], arr[j])$
        \ENDIF
        \ENDFOR
        \STATE \textbf{return} $i$
    \end{algorithmic}
\end{algorithm}

\newpage
四:
\\
1.选取两组数的第 $k / 2$ 个数k1、k2进行比较。
\\
2.若k1小于k2,则将第一组数前 $k / 2$ 个数排除,重新计算 $k$, 在剩余数中寻找。
\\
3.否则排除第二组数的前 $k / 2$ 个数排除,重新计算 $k$, 在剩余数中寻找。
\\
4.直到 $k = 1$,此时返回两组数中当前索引下的更小的一个数,即为第 $k / 2$ 大的数。

\begin{algorithm}
    \caption{\textbf{search for two arrays' k-th number}}
    \label{alg:arrays_k-th}
    \begin{algorithmic}[1]
        \REQUIRE $nums1$, $nums2$, $k$
        \ENSURE $the median of the two arrays$
        \STATE $length1 = len(nums1)$
        \STATE $length2 = len(nums2)$
        \STATE $index1 = index2 = 0$
        \WHILE{$true$}
        \IF{$index1 == index2$}
        \STATE \textbf{return} $nums2[index2 + k - 1]$
        \ENDIF
        \IF{$index2 == length2$}
        \STATE \textbf{return} $nums1[index1 + k - 1]$
        \ENDIF
        \IF{$k == 1$}
        \STATE \textbf{return} $min(nums1[index1], nums2[index2])$
        \ENDIF
        \STATE $new\_index1 = min(index1 + k / 2, length1) - 1$
        \STATE $new\_index2 = min(index2 + k / 2, length2) - 1$
        \IF{$nums1[new\_index1] \leq nums2[new\_index2]$}
        \STATE $k = k - (new\_index1 - index1 + 1)$
        \STATE $index1 = new\_index1 + 1$
        \ELSE
        \STATE $k = k - (new\_index2 - index2 + 1)$
        \STATE $index2 = new\_index2 + 1$
        \ENDIF
        \ENDWHILE
    \end{algorithmic}
\end{algorithm}

上述算法实现了在 $O(log(m + n))$ 的时间复杂度下找出两组升序排列的数中的第 $k$ 大的数,
若两组数总长度为偶数,则使用上述算法找出第 $(m + n) / 2$ 大和第 $(m + n) / 2 + 1$ 
大的数并取两者的平均值;若两组数总长度为奇数,则使用上述算法找出第 $(m + n) / 2$ 大的数即可。
时间复杂度均为 $O(log(m + n))$ 


\newpage
五:
\\
1.遍历数组,统计每个数字出现的次数,存入字典。
\\
2.使用快速排序对字典的值进行排序,得到一个递减的序列。
\\
3.选出递减序列前k个键值对的键。
\\
时间复杂度为$O(n \log n)$

\begin{algorithm}
    \caption{\textbf{top k numbers}}
    \label{alg:top_k}
    \begin{algorithmic}[1]
        \REQUIRE $nums$, $k$
        \ENSURE top k numbers
        \STATE $map = \{\}$
        \STATE $n = len(nums)$
        \FOR{$i = 0$ to $n - 1$}
        \STATE $key = nums[i]$
        \STATE $map[key] = map[key] + 1$
        \ENDFOR
        \STATE $sorted\_map = quick\_sort(map,\; map.items())$
        \STATE $return\_nums = [\;]$
        \FOR{$i = 0$ to $n - 1$}
        \IF{$i \leq k$}
        \STATE $return\_nums.append(sorted\_map[i][0])$
        \ENDIF
        \ENDFOR
        \STATE \textbf{return} $return\_nums$
    \end{algorithmic}
\end{algorithm}

\end{document}
