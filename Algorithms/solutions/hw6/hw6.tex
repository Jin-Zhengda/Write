\documentclass{article}
\usepackage{xeCJK}
\usepackage{amsmath}


\setCJKmainfont{Microsoft YaHei}
\linespread{1.5}


\begin{document}
一:\\
操作开销函数为
\[
W(i) = 
\begin{cases}
i,\; i = 2^k, k = 0, 1, 2, \cdots \\
1,\; other  
\end{cases}
\]
(1)聚集法:\\
$n$ 次操作内有 $\lfloor{\log_{2}n}\rfloor + 1$ 个数为 
$2^k, k = 0, 1, 2, \cdots$ \\
则 $n$ 次操作的总代价为
\[
\begin{aligned}
T(n) & = n - (\lfloor{\log_{2}n}\rfloor + 1) + \sum\limits_{k = 0}^{\lfloor{\log_{2}n}\rfloor + 1}2^k \\
     & \leq n - (\lfloor{\log_{2}n}\rfloor + 1) + \sum\limits_{k = 0}^{\log_2^n + 1}2^k \\
     & = 3n - \lfloor{\log_{2}n}\rfloor - 2 \\
     & = O(n) 
\end{aligned} 
\]
故平摊代价为 $T(n) / n = O(1)$。\\
(2)会计法:\\
设操作的摊还代价为:
\[
W(i) = 
\begin{cases}
0,\; i = 2^k, k = 0, 1, 2, \cdots \\
m,\; other  
\end{cases}
\]
其中 $m$ 为常数且 $m > 0$。
实际代价为 $i$ 的操作数量为 $\lfloor{\log_{2}n}\rfloor + 1$,
实际代价为 $1$ 的操作数量为$n - \lfloor{\log_{2}n}\rfloor + 1$,
故在所有操作中,需要满足
\[
m(n - \lfloor{\log_{2}n}\rfloor - 1) \geq \sum\limits_{k = 0}^{\lfloor{\log_{2}n}\rfloor + 1}2^k
\]
而
\[
m(n - \lfloor{\log_{2}n}\rfloor - 1) \geq m(n - \log_{2}n - 1)
\]
且
\[
\sum\limits_{k = 0}^{\lfloor{\log_{2}n}\rfloor + 1}2^k \leq \sum\limits_{k = 0}^{\log_{2}n + 1}2^k = 2n - 1
\]
故只需要
\[
m(n - \log_{2}n - 1) \geq 2n - 1
\]
求得 $m \geq 20$,
即在 $m \geq 20$ 时实际代价为 $1$ 的操作总能使用自己的摊还余额来支付实际操代价
为 $i$ 的操作的开销,总的摊还代价始终非负。\\
实际代价为 $1$ 的操作数量小于 $n$, 故总摊还代价小于 $mn$,时间复杂度为$O(n)$,
从而平摊代价为 $O(1)$。\\
(3)势能法:\\
定义势能函数$\phi(D)$为对该数据结构的操作次数,\\
则对于 $i = 2^k, k = 0, 1, 2, \cdots$,
实际代价 $c_i = i$,势差 $\phi(D_i) - \phi(D_{i - 1}) = 1$,从而摊还代价
$c_i^{'} = i + 1$。\\
对于 $i \neq 2^k, k = 0, 1, 2, \cdots$,
实际代价 $c_i = 1$,势差 $\phi(D_i) - \phi(D_{i - 1}) = 1$,从而摊还代价
$c_i^{'} = 2$。\\
总摊还代价为
\[
\begin{aligned}
     T(n) & = n + \sum\limits_{k = 0}^{\lfloor{\log_{2}n}\rfloor + 1}2^k \\
          & \leq n + \sum\limits_{k = 0}^{\log_2^n + 1}2^k \\
          & = 3n - 2 \\
          & = O(n) 
\end{aligned} 
\]
由于 $\phi(D_i) \geq \phi(D_0)$,故总的摊还代价为总的实际代价的一个上界,
从而平摊代价为 $T(n) / n = O(1)$。



\newpage
二:\\
若 $i \neq 2^k, k = 0, 1, \cdots$,则操作代价为 $1$,\\
若 $i = 2^k, k = 0, 1, \cdots$,则操作代价为 $i + 1$。\\
(1)聚集法:\\
$n$ 次操作内有 $\lfloor{\log_{2}n}\rfloor + 1$ 个数为 
$2^k, k = 0, 1, 2, \cdots$ \\
则 $n$ 次操作的总代价为
\[
\begin{aligned}
T(n) & = n + \sum\limits_{k = 0}^{\lfloor{\log_{2}n}\rfloor + 1}2^k \\
     & \leq n + \sum\limits_{k = 0}^{\log_2^n + 1}2^k \\
     & = 3n - 2 \\
     & = O(n) 
\end{aligned} 
\]
故平摊代价为 $T(n) / n = O(1)$。\\
(2)会计法\\
设操作的摊还代价为:
\[
W(i) = 
\begin{cases}
i + 2,\; i = 2^k, k = 0, 1, 2, \cdots \\
2,\; other  
\end{cases}
\]
其中 $m$ 为常数且 $m > 0$。
实际代价为 $i$ 的操作数量为 $\lfloor{\log_{2}n}\rfloor + 1$,
实际代价为 $1$ 的操作数量为$n - \lfloor{\log_{2}n}\rfloor + 1$,
故总代价为
\[
\begin{aligned}
     T(n) & = n + \sum\limits_{k = 0}^{\lfloor{\log_{2}n}\rfloor + 1}2^k + \lfloor{\log_{2}n}\rfloor + 1\\
          & \leq n + \sum\limits_{k = 0}^{\log_2^n + 1}2^k + \log_2n + 1\\
          & = 3n + \log_2n + 1 \\
          & = O(n) 
\end{aligned} 
\]
从而平摊代价为 $T(n) / n = O(1)$\\
(3)势能法\\
定义势能函数$\phi(D)$为,\\
则对于 $i = 2^k, k = 0, 1, 2, \cdots$,
实际代价 $c_i = i + 1$,势差 $\phi(D_i) - \phi(D_{i - 1}) = 1$,从而摊还代价
$c_i^{'} = i + 2$。\\
对于 $i \neq 2^k, k = 0, 1, 2, \cdots$,
实际代价 $c_i = 1$,势差 $\phi(D_i) - \phi(D_{i - 1}) = 1$,从而摊还代价
$c_i^{'} = 2$。\\
总摊还代价为
\[
\begin{aligned}
     T(n) & = n + \sum\limits_{k = 0}^{\lfloor{\log_{2}n}\rfloor + 1}2^k + \lfloor{\log_{2}n}\rfloor + 1\\
          & \leq n + \sum\limits_{k = 0}^{\log_2^n + 1}2^k + \log_2n + 1\\
          & = 3n + \log_2n + 1 \\
          & = O(n) 
\end{aligned} 
\]
由于 $\phi(D_i) \geq \phi(D_0)$,故总的摊还代价为总的实际代价的一个上界,
从而平摊代价为 $T(n) / n = O(1)$。

\end{document}