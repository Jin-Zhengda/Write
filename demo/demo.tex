\documentclass{article}
\usepackage[normalem]{ulem}
\usepackage{xeCJK}
\usepackage{indentfirst}
\usepackage{setspace}
\usepackage{marginnote}
\usepackage{hyperref}
\usepackage{fontspec}
\usepackage{amsmath}


\hypersetup{
    colorlinks=true,
    linkcolor=blue, % 将链接颜色更改为蓝色
    urlcolor=blue,  % 将URL的颜色也更改为蓝色
}
\title{Latex Notes}
\author{Jin}
\date{\today}
\setcounter{tocdepth}{2}
\setlength{\parindent}{2em} %段落首行缩进距离
\addtolength{\parskip}{3pt} %段落间距
\linespread{1.2} %设置行距

\setCJKmainfont{SimSun}
\setCJKmonofont{Microsoft YaHei}
\setmainfont[Mapping=tex-text]{Times New Roman}
\setsansfont[Mapping=tex-text]{Tahoma}
\setmonofont{Courier New}

\renewcommand{\labelitemi}{-} % 改变列表符号及编号形式
\renewcommand{\theenumi}{\alph{enumi}}

\begin{document}
\maketitle
\tableofcontents % 生成目录

This is document.
\section{Section 1}
\Large{This} \small{is} section 1.

\subsection{Subsection 1.1}
This is subsection 1.1. \\
\# \{ \} \~ \textbackslash \\
\LaTeX \\
\v{A} \={A} \~{A} \textcircled{A}

\subsubsection{Subsubsection 1.1.1}
\textit{This is subsubsection 1.1.1.}  
\begin{spacing}{1.3}
    \emph{This is emphesis.} \\
    \uline{underline} \\
    \uwave{waveline} \\
    \sout{strike-out}
\end{spacing}




\newpage
\section{Section 2}

\paragraph{Paragraph}

\begin{flushleft}
    This is paragraph.
\end{flushleft}
\begin{quote}
引文两端\\都缩进。
\end{quote}

\begin{quotation}
引文两端缩进,首\\行增加缩进。
\end{quotation}

\begin{verse}
引文两端缩进,第二行\\起增加缩进。
\end{verse}

\verb|command| 行 间 命 令
\begin{verbatim}
printf("Hello   , world!");
\end{verbatim}
% 可以标出空格
\begin{verbatim*} 
printf("Hello   , world!"); 
\end{verbatim*}

\begin{center}
    This is the center of the page.
\end{center}

正文\footnote{脚注}

\marginnote{正常边注}
\reversemarginpar
\marginnote{反向边注}
\normalmarginpar

\subparagraph{Subparagraph} 
This is subparagraph.

\begin{itemize}
    \item C++
    \item Java
    \item HTML
\end{itemize}

\begin{enumerate}
    \item C++
    \item Java
    \item HTML
\end{enumerate}

一个标签\label{marker}

\begin{description}
    \item[C++] 编 程 语 言
    \item[Java] 编 程 语 言
    \item[HTML] 标 记 语 言
\end{description}

\newpage
\mbox{010 6278 5001}
\fbox{010 6278 5001}

%语法:[宽度][对齐方式]{内容}
\makebox[100pt][c]{仪仗队}

\framebox[100pt][s]{仪仗队}


% 语法:[外部对齐][高度][内部对齐]{宽度}{内容}
\fbox{
 \parbox[c][36pt][t]{140pt}{
锦瑟无端五十弦,一弦一柱思华年。庄生晓梦迷蝴蝶,望帝春心托杜鹃。
 }
}

\hfill

\fbox{
 \begin{minipage}[c][36pt][b]{140pt}
沧海月明珠有泪,蓝田日暖玉生烟。此情可待成追忆,只是当时已惘然。
\end{minipage}
}

第\pageref{marker}页\ref{marker}节

\begin{equation}
    E=mc^2 \label{eq:einstein}
\end{equation}

As derived in Equation~\ref{eq:einstein}

Einstein's $E=mc^2$
\[ E=mc^2 \]
\[ \boxed{E=mc^2} \]
\begin{equation}
E=mc^2
\end{equation}

\[
x_{ij}^2\quad \sqrt{x}\quad \sqrt[3]{x} 
\]

$ \frac{1}{2} \dfrac{1}{2} $
\[ \frac{1}{2}
\tfrac{1}{2} \]


\[ \pm\; \times\; \div\; \cdot\; \cap\; \cup\;
\geq\; \leq\; \neq\; \approx\; \equiv \]


\newpage
$ \sum_{i=1}^n i\quad \prod_{i=1}^n\quad
\lim_{x\to0}x^2\quad \int_a^b x^2 dx $\\
$ \sum\limits_{i=1}^n i\quad \prod\limits_{i=1}^n\quad
\lim\limits_{x\to0}x^2\quad \int\limits_a^b x^2 dx $
\[ \sum_{i=1}^n i\quad \prod_{i=1}^n\quad
\lim_{x\to0}x^2\quad \int_a^b x^2 dx \]
\[ \sum\nolimits_{i=1}^n i\quad
\prod\nolimits_{i=1}^n\quad
\lim\nolimits_{x\to0}x^2\quad
10 \int\nolimits_a^b x^2 dx \]


\[ \int\int\quad \int\int\int\quad
\int\int\int\int\quad \int\dots\int \]
\[ \iint\quad \iiint\quad \iiiint\quad \idotsint \]

\[ \xleftarrow{x+y+z}\quad
\xrightarrow[x<y]{a*b*c} \]

\[
\bar{x} \quad \hat{x} \quad \vec{x} \\
\overline{xxx} \quad \underline{xxx} \quad \overbrace{xxx} 
\quad \underbrace{xxx} \quad \overrightarrow{xxx}
\]

\[ \Bigg(\bigg(\Big(\big((x)\big)\Big)\bigg)\Bigg)\quad
\Bigg[\bigg[\Big[\big[[x]\big]\Big]\bigg]\Bigg]\quad
\Bigg\{\bigg\{\Big\{\big\{\{x\}\big\}\Big\}\bigg\}\Bigg\}
\]\[
\Bigg\langle\bigg\langle\Big\langle\big\langle\langle x
\rangle\big\rangle\Big\rangle\bigg\rangle\Bigg\rangle\quad
\Bigg\lvert\bigg\lvert\Big\lvert\big\lvert\lvert x
\rvert\big\rvert\Big\rvert\bigg\rvert\Bigg\rvert\quad
\Bigg\lVert\bigg\lVert\Big\lVert\big\lVert\lVert x
\rVert\big\rVert\Big\rVert\bigg\rVert\Bigg\rVert \]

\[ x_1,x_2,\dots,x_n\quad 1,2,\cdots,n\quad
\vdots\quad \ddots \]

a\, a\: a\; a\quad a\qquad a\! a

\[ \begin{pmatrix} a&b\\c&d \end{pmatrix} \quad
2 \begin{bmatrix} a&b\\c&d \end{bmatrix} \quad
\begin{Bmatrix} a&b\\c&d \end{Bmatrix} \quad
4 \begin{vmatrix} a&b\\c&d \end{vmatrix} \quad
\begin{Vmatrix} a&b\\c&d \end{Vmatrix} \]

Marry has a little matrix $ ( \begin{smallmatrix}
    a&b\\c&d \end{smallmatrix} ) $.

\begin{multline}
    x = a+b+c+ \\
    d+e+f+g
\end{multline}

\[ \begin{split}
    x = &a+b+c+ \\
    &d+e+f+g
\end{split} \]

\begin{gather}
    a = b+c+d \\
    x = y+z
\end{gather}

\begin{align}
    a &= b+c+d \\
    x &= y+z
\end{align}

\[ 
y=
\begin{cases}
    -x,\quad x\leq 0 \\ 
    x,\quad x>0
\end{cases} 
\]



\end{document}