\documentclass{article}
\usepackage[normalem]{ulem}
\usepackage{xeCJK}
\usepackage{indentfirst}
\usepackage{setspace}
\usepackage{marginnote}

\title{Latex Notes}
\author{Jin}
\date{\today}
\setcounter{tocdepth}{2}
\setlength{\parindent}{2em} %段落首行缩进距离
\addtolength{\parskip}{3pt} %段落间距
\linespread{1.2} %设置行距
\setCJKmainfont{SimSun}
\renewcommand{\labelitemi}{-} % 改变列表符号及编号形式
\renewcommand{\theenumi}{\alph{enumi}}

\begin{document}
\maketitle
\tableofcontents % 生成目录

This is document.
\section{Section 1}
\Large{This} \small{is} section 1.

\subsection{Subsection 1.1}
This is subsection 1.1. \\
\# \{ \} \~ \textbackslash \\
\LaTeX \\
\v{A} \={A} \~{A} \textcircled{A}

\subsubsection{Subsubsection 1.1.1}
\textit{This is subsubsection 1.1.1.}  
\begin{spacing}{1.3}
    \emph{This is emphesis.} \\
    \uline{underline} \\
    \uwave{waveline} \\
    \sout{strike-out}
\end{spacing}

一个标签\label{marker}


\newpage
\section{Section 2}

\paragraph{Paragraph}

\begin{flushleft}
    This is paragraph.
\end{flushleft}
\begin{quote}
引文两端\\都缩进。
\end{quote}

\begin{quotation}
引文两端缩进,首\\行增加缩进。
\end{quotation}

\begin{verse}
引文两端缩进,第二行\\起增加缩进。
\end{verse}

\verb|command| 行 间 命 令
\begin{verbatim}
printf("Hello   , world!");
\end{verbatim}
% 可以标出空格
\begin{verbatim*} 
printf("Hello   , world!"); 
\end{verbatim*}

\begin{center}
    This is the center of the page.
\end{center}

正文\footnote{脚注}

\marginnote{正常边注}
\reversemarginpar
\marginnote{反向边注}
\normalmarginpar

\subparagraph{Subparagraph} 
This is subparagraph.

\begin{itemize}
    \item C++
    \item Java
    \item HTML
\end{itemize}

\begin{enumerate}
    \item C++
    \item Java
    \item HTML
\end{enumerate}

\begin{description}
    \item[C++] 编 程 语 言
    \item[Java] 编 程 语 言
    \item[HTML] 标 记 语 言
\end{description}

\newpage
\mbox{010 6278 5001}
\fbox{010 6278 5001}

%语法:[宽度][对齐方式]{内容}
\makebox[100pt][c]{仪仗队}

\framebox[100pt][s]{仪仗队}


% 语法:[外部对齐][高度][内部对齐]{宽度}{内容}
\fbox{
 \parbox[c][36pt][t]{140pt}{
锦瑟无端五十弦,一弦一柱思华年。庄生晓梦迷蝴蝶,望帝春心托杜鹃。
 }
}

\hfill

\fbox{
 \begin{minipage}[c][36pt][b]{140pt}
沧海月明珠有泪,蓝田日暖玉生烟。此情可待成追忆,只是当时已惘然。
\end{minipage}
}



第\pageref{marker}页\ref{marker}节


\end{document}